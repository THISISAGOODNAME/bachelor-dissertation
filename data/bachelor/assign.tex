% !Mode:: "TeX:UTF-8"
% 任务书中的信息
%% 原始资料及设计要求
\assignReq
{原始资料: lua源代码}
{原始资料: SQLite源代码}
{原始资料: box2d源代码}
{原始资料: bullet physics源代码}
{原始资料: VCGlibrary以及meshlab源代码}
{设计要求: 将上述软件功能移植到web}
%% 工作内容
\assignWork
{移植lua解释器/虚拟机到html5,在前端运行lua代码}
{移植SQLite到html5,在前端执行一个建议的数据库}
{移植box2d到html5,得到一个2维物理引擎}
{移植bullet physics到html5,得到一个能在webgl环境使用的3维物理引擎}
{移植VCLlibrary,得到一个在线的网格处理框架,并尝试移植meshlab的UI到web上}
%% 参考文献 (函数最多9个参数,多余9篇文献只能另写函数)
\assignRef
{从 Objective-C 到 Swift 的软件移植研究与实现}
{Essential JNI: Java Native Interface}
{Google Native Client: The web of the future—or the past?}
{Emscripten: an LLVM-to-JavaScript compiler}
{LLVM: A compilation framework for lifelong program analysis \& transformation}
{asm. js: Working Draft 17 March 2013}
{2.0 for the Web}
{LLVM and Clang: Next generation compiler technology}
{language family frontend for LLVM}
\assignRefTwo
{GNU Make: A program for directing recompilation, for version 3.81}
{Collection makefile generator}
{Web IDL}
{Meshlab: an open-source 3d mesh processing system}
{Optimization of dynamic languages using hierarchical layering of virtual machines}
\assignRefThree
{Bullet physics engine}
{Application research of embedded database SQLite}
{浅谈不同平台的软件移植研究}
{软件移植理论与技术研究}
{硬件异构平台下应用软件移植}
