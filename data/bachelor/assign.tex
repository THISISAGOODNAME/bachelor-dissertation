% !Mode:: "TeX:UTF-8"
% 任务书中的信息
%% 原始资料及设计要求
\assignReq
{1 原始资料: lua源代码}
{2 原始资料: SQLite源代码}
{3 原始资料: box2d源代码}
{4 原始资料: bullet physics源代码}
{5 原始资料: emscripten和clang/llvm的源代码}
{6 设计要求: 将上述软件功能移植到web}
%% 工作内容
\assignWork
{1 移植lua解释器/虚拟机到html5,在前端运行lua代码}
{2 移植SQLite到html5,在前端执行一个建议的数据库}
{3 移植box2d到html5,得到一个2维物理引擎}
{4 移植bullet physics到html5,得到一个能在webgl环境使用的3维物理引擎}
{5 剖析emscripten的结构,功能以及用法}
%% 参考文献 (函数最多9个参数,多余9篇文献只能另写函数)
\assignRef
{1 从 Objective-C 到 Swift 的软件移植研究与实现}
{2 Essential JNI: Java Native Interface}
{3 Google Native Client: The web of the future—or the past?}
{4 Emscripten: an LLVM-to-JavaScript compiler}
{5 LLVM: A compilation framework for lifelong program analysis \& transformation}
{6 asm. js: Working Draft 17 March 2013}
{7 2.0 for the Web}
{8 LLVM and Clang: Next generation compiler technology}
{9 language family frontend for LLVM}
\assignRefTwo
{10 GNU Make: A program for directing recompilation, for version 3.81}
{11 Collection makefile generator}
{12 Web IDL}
{13 Meshlab: an open-source 3d mesh processing system}
{14 Optimization of dynamic languages using hierarchical layering of virtual machines}
\assignRefThree
{15 Bullet physics engine}
{16 Application research of embedded database SQLite}
{17 浅谈不同平台的软件移植研究}
{18 软件移植理论与技术研究}
{19 硬件异构平台下应用软件移植}
