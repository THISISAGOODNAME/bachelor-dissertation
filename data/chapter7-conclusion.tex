% !Mode:: "TeX:UTF-8"
\chapter{总结与展望}

\par 

\indent \indent 伴随着硬件技术的超高速发展,移动设备的性能井喷一样的增长。在2007年,给你这么一部手机,CPU主频416MHz,内存128MB,拥有3.5寸320*480的分辨率的全触摸屏,可以使用GSM访问互联网,内置蓝牙和重力感应,整个世界都为之沸腾了,没错,这就是初代iPhone。这手机配置放在现在,也许放桌子上都没人偷,手机还没手机卡值钱。而现在,4核心CPU已经成为入门手机的标配,8核乃至10核的移动CPU也不是没有。移动硬件用短短几年的时间做到了过去桌面硬件十几年才走完的路。现在的移动设备的处理能力已经可以赶上PC的性能。

在移动设备的硬件性变得能越来越强悍的同时,移动操作系统的竞争也基本达到了尾声。封闭但变得日益开放的iOS系统,开放但版本混乱四分五裂的Android系统以及市场反响平平的Windows Mobile。仅仅只iOS和android两个平台,就非常让那些习惯了在PC的传统软件开发者以及企业头疼。

但是,与此同时,我们也看到HTML 5的能力不断变强大,Web App和Native App之争也经历了数次。在2011、2012年的时候,html 5 + phonegap的组合确实因为不够成熟遭受了很大质疑,不过在W3C HTML5标准彻底定稿之后,HTML5技术又迎来了告诉发展期。Ionic、Weex等更为成熟的html 5 hybrid APP框架的出现,使 html 5 跨平台APP重新回到人们的视线,甚至还出现了nw.js,electron以及libCEF这些使用HTML5技术开发桌面应用的方法。微信公众号平台对于HTML5的推广也起到了很大的推动作用。
无论如何,当移动设备的硬件发展到一定程度,web app的性能问题得到改善之后,快速开发程序的能力以及构建程序的成本在技术选型时的权重会越来越高。

从开发者的角度,我是Web App的坚定支持者和重度用户,Web App相比Native App有两个显著的优势:开发难度低、跨平台能力强。Web App也确实越来越受开发者青睐。腾讯的微信还有阿里巴巴的淘宝客户端已经新浪微博手机客户端,都是使用HTML5做前端,腾讯的微信更是自豪的写出“由QQ浏览器X5提供技术支持”。

从普通用户的角度,Web App可以省去频繁更新的麻烦,虽然可能会多费一点流量。而且用户无需自己手动更新不代表Web App不会进行更新,只是更新的地方从客户端转移到了服务器端,开发人员可以自行控制迭代,无需经过苹果app store复杂而缓慢的审核流程,用户每次打开,得到的都是该应用的最新版本,获取的永远是最新的体验。而且,web app甚至无需安装,打开浏览器直接就可以运行。Web App的运行环境,本身就深深的扎根于浏览器之中,而浏览器又是所有智能手机都自带的重要App,无论android还是iOS都将webkit作为自带浏览器的内核,统一浏览器内核的好处就是高兼容性。Web App开发只要考虑webkit浏览器,无须考虑平台差异,一款产品就能满足几乎平台的需求,还能显著降低开发成本和开发难度。

HTML5是标准化的,有统一的标准,保证了Web App的兼容性和使用体验一致性。因此Web App的另一个很大的优势是有非常卓越的互联互通特性。7、8年前,flash页游席卷全国,靠的就是flash自己的互联互通的特性。现在,网页游戏,以微信公众号平台为依托,再次流行起来,也是如此。Web App可以轻松的吸引流量,同时,他还能把流量输出给其他的页面,导流引流。卓越的流量导流特性,对于互联网产品的推广方式和营销策略都产生了深远的影响。

过去,受限于各种开发工具、调试工具、开发框架和性能需求,大量开发者选择了C/C++作为首选开发语言。随着物联网的兴起和移动网络的繁荣,移动化、web化已经成为趋势。许多应用在未来或者现在都将面临从C++到html5的移植问题。

本文通过工具emscripten的剖析和一些具体案例,展示了黑箱程序和类库程序两大类传统软件向html5移植的方法,并通过实例,检验了移植后项目的可用性、可行性。传统软件通过emscripten向web移植的潜力,还有很大的挖掘空间。我也希望我的工作能够对于后来的人,产生启发和帮助。

我本来还想移植meshlab和VCGLibrary库到web上,但是受限于个人能力,以及本人对VCGLibrary的不了解,并没有移植这个经典的网格处理库。对于Qt开发的不熟悉,以及Qt程序的特点,Qt程序的界面往往不是C++语言而是Qt widget或者Qt QML。对这些框架的陌生让我并没有完成这些工作。也希望有来者,能够完成这些事情。