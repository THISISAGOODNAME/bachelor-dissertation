% !Mode:: "TeX:UTF-8"

% 学院中英文名,中文不需要“学院”二字
% 院系英文名可从以下导航页面进入各个学院的主页查看
% http://www.buaa.edu.cn/xyykc/index.htm
\school
{计算机}{School of Computer Science and Engineering}

% 专业中英文名
\major
{计算机科学与技术}{Computer Science and Technology Engineering}

% 论文中英文标题 以luaVM、SQLite、Bullet Physics 以及 meshlab 为例
\thesistitle
{基于 Emscripten 的传统软件 Web 移植}
{}
{Old software migration based on emscripten}
{}

% Take luaVM, SQLite, Bullet Physics and meshlab as examples

% 作者中英文名
\thesisauthor 杨彦君 Yanjun Yang
{}{}

% 导师中英文名 吴壮志 Zhuangzhi WU
\teacher
{}{}
% 副导师中英文名
% 注:慎用‘副导师’,见北航研究生毕业论文规范
%\subteacher{副导师}{subteacher}

% 中图分类号,可在 http://www.ztflh.com/ 查询
\category{TP391.7}

% 本科生为毕设开始时间;研究生为学习开始时间
\thesisbegin{2015}{12}{29}

% 本科生为毕设结束时间;研究生为学习结束时间
\thesisend{2016}{07}{01}

% 毕设答辩时间
\defense{2016}{06}{XX}

% 中文摘要关键字
\ckeyword{移植,emscripten,web}

% 英文摘要关键字
\ekeyword{migration, emscripten, web}
