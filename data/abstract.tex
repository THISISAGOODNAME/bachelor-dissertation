% !Mode:: "TeX:UTF-8"

% 中英文摘要
\begin{cabstract}
面对人们日益加剧的信息分享和获取的需求,硬件方面智能终端设备也在不
断的推陈出新,而同样的,在软件方面,移动互联网也逐渐渗透到各行各业当中,
产生了“互联网+”和“工业 4.0”的概念。全世界进入了移动互联网的时代。
而在移动互联网时代,大量基于 C/S 的传统软件变为了遗留系统。
这些遗留系统在向web环境进行移植过程中,需要花费巨大的人力和物力,
如何有效的进行软件移植工作,一直存在普遍的问题。

\par
作者完成的主要工作如下:

\begin{enumerate}
    \item 移植lua解释器/虚拟机到html5,在前端运行lua代码
    \item 移植SQLite到html5,在前端执行一个简易的数据库
    \item 移植box2d到html5,得到一个2维物理引擎
    \item 移植bullet physics到html5,得到一个能在webgl环境使用的3维物理引擎
    \item 移植VCLlibrary,得到一个在线的网格处理框架,并尝试移植meshlab的UI到web上
\end{enumerate}

\par
通过工程实际,证实传统软件直接进行web/html5移植的可能性与可行性。
并设计工作流,简化、系统化移植工作。

\end{cabstract}

\begin{eabstract}
With the requirement of information sharing and access,
intelligent devices are also in gaint innovation.
The concept, "the Internet plus" and "industry 4.0",
is the main stream of the time.The world has entered the era of mobile internet.
However, there are still many legacy software based on the C/S architecture.
Porting these software may cost a lot labor forces and financial forces.
It is a problem that how to migrate these legacy software.

\par
The main research of author in this thesis is summarized as follows:

\begin{enumerate}
    \item Porting the Lua interpreter / virtual machine to HTML5, running Lua code on browser
    \item Porting SQLite to HTML5, run a simple database on browser
    \item Porting box2d to HTML5 to get a 2D physics engine
    \item Transplant physics bullet to HTML5, to get a 3 dimensional physics engine that can be used in the webgl environment
    \item Transplant VCLlibrary, get an online mesh processing framework, and try to transplant meshlab UI to web
\end{enumerate}

\par
Through the practical engineering,comfirmd the possibility and feasibility
of traditional software transporting to web/html5 directly.
And the design of workflow, simplified the assignment of transplantation.
\end{eabstract}